%!TEX ROOT = thesis.tex
\chapter{Literature Review}
\section{Introduction}
The chapter will be discussing the existing literature of ANNs methods and its usage. The applications in which the methods have been applied and the results have mentioned in this chapter.Artificial Neural Networks (ANNs) is the recent revival research area which has started since 1940's. ANNs is the idea of construction  a computable machine which mimics the biological brain structure and the processes of the human brain. ANNs is one of the widely used methods used  in forecasting not only in the financial market, as well in many different areas of research.

\section{Literature Review}

During recent decades, many kinds of literature have been developed in the area of neural networks methods. Many applications apply ANNs methods  to forecast in different fields such as medical diagnostics, general business, financial markets and so on. ANNs methods  are popular because of the ability to generalize and learn from training process or experiences. Forecasting is one of the first successful areas which applied ANNs techniques which is also known as nonlinear techniques.  

Varieties of research show that ANNs methods  produce better results in different applications compared to traditional forecasting methods. The  research which was done by \citeA{ochi:2002}  has shown that  ANNs produce the effective  results on  the prediction of survival rates for patients with uterine cervical cancer. 

Besides, \citeA{maqsood:2004} has also shown that ANNs produces reliable accurate performance while forecasting the weather compared to statistical regression models. Moreover, \citeA{gao:2010} attempted to forecast the water consumption using ANNs methods. The research showed that ANNs methods perform higher forecasting rate and accuracy.

ANNs methods are widely used in forecasting exchange rates. Recent researches have shown  significant results in exchange rates forecasting using ANNs methods.\citeA{philip:2011} have explained the superiority of ANNs methods  compared to traditional statistical methods. They assembled feedforward ANNs network to forecast exchange rates  using MLPs, backpropagation algorithm as training algorithm, and Sigmoid function as the activation function. Their network showed that  it had an accuracy of 81.2\% in forecasting exchange rates.

\citeA{chan:1995} designed a simple neural network to improve regular technical analyses which have in traditional forecasting techniques. In their findings, a neural network is also able to adapt itself to new patterns emerging in the marketplace. This is significant finding because the change in currency market conditions is very rapidly.

In the paper of \citeA{pacelli:2011}, they made two hypotheses to valid using ANNs networks. They tried to validate that : the financial markets pricing process  is not random, and the degree of information efficiency of the financial markets is not strong or semi-strong. Their ANNs was built by using MLPs methods. Their research found that the dependence of prices in financial markets is not completely random, and therefore, it can be predictable.

In their paper \citeA{hua:2010}, to predict the exchange rate, they used ANNs methods, and Kernel Regression methods, to smooth the noise. The research tried to predict for the three different exchange rates, one, three, and twelve months ahead by using training data sets of US to British Pound exchange rates, Indian Rupees, and Japanese Yen. They compared their methods to traditional statistical forecasting methods and found that their ANNs with Kernel Regression methods outperformed the traditional methods.

ANNs methods outperform other methods not only in exchange rate forecasting but also in the stock market prediction. According to \citeA{cao:2005}, their research has shown that ANNs methods outperform linear models forecasting methods. They suggested that ANNs are useful tools for prediction stock market in emerging markets like China.

\citeA{sermpinis:2012} found in their paper that their forecasting is improved by using RNN, MLP methods which are the widely used  ANNs methods. Moreover, according to \citeA{cao:2005}, one of the ANNs feed forward network, MLPs outperform other techniques when it comes to financial time series prediction. It is also the most used neural network for time series forecasting. 

Radial basis function neural networks (RBF)are also applied in different fields. \citeA{sermpinis:2013} has done a study which forecasted the exchange rates by using RBF neural network functions. It is shown that the neural networks methods produced better results compared to statistical linear models.

All the above paper  constructed by using single ANN methods or technique. It has been shown the accuracy of the results is much better with  ensemble ANNs methods. Ensemble ANNs networks are constructed by combining two or more ANNs methods, or by applying different kinds of learning algorithms to train the dataset, or using the two or more same  ANNs network as a combination network.

It the paper, \citeA{hansen:1992}   showed that simple ensembles can perform better generalizations than a single ANN and that a subset of all possible ANNs can achieve better results than a single ANN. Moreover, \citeA{giacomel:2015} constructed the ensemble ANNs network by using two ANNs  architectures  with a different number of nodes, input layers, hidden layers, and outputs layers to predict financial time series market. The research shows that the ensemble ANNs  network outperformed the single ANN network model. Their ensemble has shown itself not as a tool that always gets the best profits, but instead gives good and consistent results.

As for the Malaysian Ringgits and USD currency exchange rates, \citeA{chan:2010} conducted a reasearch on whether using ANNs model will give the desirable accuracy outputs. In their paper, the results mentioned that ANNs models outperformed the random walks model and, produced the desirable accuracy for the prediction with is 0.02065 RMSE error rate.

Besides, \citeA{adhikari:2013} showed in their paper that accuracies of the ANN forecasting are significantly better and improved  through ensemble method. It is  showed that this combined training algorithm performs much better than each of the individual ones. However, methodologies based on single ANN method have been largely used for financial time series prediction while ANNs ensembles are still little used in this area. Hence, this project plans to apply ensembles ANNs methods to forecast the exchange rates.

Ensemble ANNs methods produce  better performance not only in forecasting the exchange rates but also in the different research fields. \citeA{shao:2014} built an ensemble ANNs network for fault diagnosis of proton exchange membrane fuel cell system. It showed that this method can improve the accuracy to 93.24\%, and greatly reduce failure  to report in fault diagnosis, while the accuracies of single  ANN network applied to  have  only the interval from 75.24\% to 85.62\%.

Moreover,\citeA{linares:2013} also applied ensemble methods for estimating global solar radiation from Meteosat satellite images by constructing as an ensemble of five optimized, MLP feedforward neural networks. The research showed that it produced accurate estimates of daily solar radiation at training and testing stations, and the model outperformed previous solar radiation estimates techniques.

ANNs ensemble methods have shown  as an effective method for rainfall forecasting. In their paper \citeA{nagahamulla:2012} constructed an ensemble ANNs network with a combination of Multilayer Feed Forward Network with Back Propagation Algorithm (BPN), Radial Basis Function Network (RBFN) and General Regression Neural Network (GRNN). The results have shown that ensemble model is better than other models for rainfall forecasting. The ensemble ANNs model predicts rainfall more accurately than individual BPN, RBFN, and GRNN models.

\section{Conclusion}
This project is proposed to use ensemble ANNs model for the forecasting the exchange rates from the extensive review of the related papers. From the extensive review, it shows that many types of research have applied single ANN method for forecasting the exchange rates, and very few applied an ensemble ANNs model which has significant accuracy. Therefore, this project will design an ensemble ANN model and test the performance for exchange rate forecasting.
